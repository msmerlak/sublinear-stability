\documentclass[12pt]{article}
\usepackage[utf8]{inputenc}
\usepackage{graphicx}
\usepackage{wrapfig}
\usepackage[margin = 3cm]{geometry}
\usepackage[safeinputenc, maxnames=10, backend = biber, sorting=none, url = false, doi = false, isbn = false, citestyle = numeric-comp]{biblatex}
\AtEveryBibitem{\clearlist{language}}
\AtEveryBibitem{\clearfield{note}}
\AtEveryBibitem{\clearfield{month}}
\AtEveryBibitem{\clearfield{publisher}}
\addbibresource{references.bib}
\renewbibmacro{in:}{}
\usepackage{colortbl}

\usepackage{amsmath}
\usepackage{amsfonts}


\title{A large complex ecosystem will be stable:\\ sublinear growth scaling and the principle of competitive coexistence}


\begin{document}


\maketitle

\begin{abstract}
% A longstanding puzzle in ecology concerns the relationship between complexity and stability in large ecosystems comprised of many interacting species. On the one hand, theoretical arguments dating back to May suggest that richer, more complex ecosystems are less stable and more prone to competitive exclusion; on the other hand, empirical observation points to the opposite relationship. Here we show that a recently discovered ecological scaling law offers a possible resolution to this puzzle. We exhibit a simple set of competition equations in which stability is enhanced by species richness.
\end{abstract}


\section{Results}

\subsection{Model}

We consider the dynamics of $S$ species whose abundances $\mathbf{n} = (n_1, \cdots , n_S)$ evolve in time according to
\begin{equation}\label{model}
    \frac{dn_i}{dt} = P(n_i) - \sum_{j\neq i}A_{ij} n_i n_j,
\end{equation}
where $P(n_i)$ is the productivity of population $i$ and the coefficients $A_{ij}$ characterize the strength of the interaction between $i$ and other populations $j$; the overall minus is chosen so that competition (our focus in this paper) corresponds to positive $A_{ij}$. 

We consider two different kinds of production functions:
\begin{equation}
    P(n_i) = \begin{cases}
        rn_i(1-n_i/K)\quad&\text{(logistic)}\\
        rn_i^k\quad&\text{(sublinear)}
     \end{cases}
\end{equation}
(We assume for simplicity that all species have the same growth rate $r$ and, in the logistic case, the same carrying capacity $K$. We also choose units such that $r=1$.) Eq. \eqref{model} with logistic production functions has been intensely studied under the name ``generalized Lotka-Volterra model". All populations are individually stable in this model: each $n_i$ converges to $n_i^* = K$ and returns to this value after a perturbation. However, multi-species equilibria are destabilized by interactions. In particular, one finds (SI) that, if interaction coefficients $A_{ij}$ are drawn independently from a distribution with mean $\mu$ and variance $\sigma^2$,\footnote{If $A_{ij}$ is drawn from a distribution with mean $m$ and variance $s^2$ and then set to zero with probability $c$---corresponding to an interaction network with connectance $c$---then $\mu = mc$ and $\sigma^2 = c(s^2 + (1-c)m^2)$.} a competitive equilibrium of the GLV model can only be stable if 
\begin{equation}\label{may}
    \sigma\sqrt{S} + \mu < 1/K.
\end{equation}

May arrived at this conclusion (in the special case $\mu = 0$, $K=1$) by assuming a  community matrix with random off-diagonal elements and fixed diagonal---the kind of random matrices which satisfy the celebrated `circular law'. This assumption does not hold in the GLV model: the elements of the community matrix $C = -\textrm{diag}(\mathbf{n}^*)(A + I/K)$ are correlated by the diagonal prefactor $D$, and as a result its spectrum is not in general shaped like a disk. Nevertheless, the condition \eqref{may} turns out to be identical to the condition that the random matrix $B=-(A + I/K)$ have eigenvalues with negative real parts. 

The generalized May inequality \eqref{may} implies that logistic intra-specific regulation must always be stronger than inter-specific competition for a community to be stable ($\mu < 1/K$). Moreover, the slightest amount of heterogeneity $\sigma>0$ implies a maximal species richness at given $(\mu, \sigma)$. These counter-intuitive properties of the GLV model are the mathematical foundations for the principle that `complexity begets instability'. 


\subsection{Linear stability: uniform interactions}

In this paper, we consider a sublinear production function with growth scaling exponent $k < 1$ instead of the usual logistic function. In this new model---as in the GLV model---the per-capita productivity $P(n)/n$ decreases with density $n$. The difference, however, is that $P(n)/n$ does not have a zero at finite $n$, implying that populations will grow to unbounded densities in the absence of competitors. Because individually species do not individually self-regulate, it would seem that competitive communities are even less likely to remain stable in this model than in the GLV. We now show that the opposite is true. 

We begin with the simplest case where all interactions have the same strength $\mu >0 $. An exact calculation (SI) shows that the community matrix has dominant eigenvalue proportional to $1 - (1-k)(S-1)$, hence the equilibrium is linearly stable iff
\begin{equation}
    S > 1 + \frac{1}{1-k}.
\end{equation}
That is, linear stability in the sublinear model is $(i)$ possible in the absence of self-regulation, $(ii)$ independent of the strength of competition, $(iii)$ enhanced by a larger species richness (in particular, a larger $S$ implies a faster return to equilibrium).\footnote{This remains true when productions drops to zero below $n_0 = 1$; the only difference is that $\mu(S-1) < r$ must hold for the system to be feasible.} Thus, in spite of the superficial similarity between sublinear and logistic production, these findings paint a picture of the complexity-stability relationship that is diametrically different than the usual one. 

\subsection{Linear stability: random interactions}

These results generalize to random interactions. In SI we show using methods developed in \cite{ahmadian_properties_2015, stone_feasibility_2018} that an equilibrium point $\mathbf{n}^*$ of the sublinear model will be linearly stable if
\begin{equation}\label{stability-condition}
    \sum_{i=1}^S\left(\frac{1}{\mu - (1-k) (n_i^*)^{k-2}}\right)^2 > \frac{1}{\sigma^2}.
\end{equation}
Using $(n_i^*)^{k-2} = \sum_j A_{ij}n_j^*/n_i^*= \mathcal{O}(\mu S)$, this condition gives, roughly, $\sigma \leq (1-k)\sqrt{S}\mu$, and therefore always holds for sufficiently large $S$ provided $\mu > 0$, leaving feasibility as the only constraint on stable coexistence. 

\subsection{Discussion: the problem with May}

Why does May's argument fail so spectacularly in competition models with sublinear production? The answer is that, contrary to May's central assumption, the diagonal elements of the community matrix are not independent of $S$; we have, rather,
\begin{equation}
    C = - \textrm{diag}(\mathbf{n}^*)A - (1-k)\textrm{diag}(A\mathbf{n}^*).
\end{equation}
The second term in this expression scales like $\mathcal{O}(\mu S)$ and therefore dominates the first term when $S$ becomes large. When $\mu > 0$ (competitive interactions), these effective self-interactions are stabilizing, resulting in stable equilibria at large $S$. 

\subsection{General stability}

The linear stability of equilibria is not the only relevant aspect of competitive dynamics; indeed feasibility (the condition that \eqref{model} admits an equilibrium in which all abundances are positive) has been noted to be more restrictive than linear stability \cite{stone_feasibility_2018}, and this is also what is found under sublinear scaling. McCann defines `general stability' as the condition that all populations regulate themselves away from excessively small or large densities \cite{mccann_diversity_2000}. Here we measure diversity by the Shannon index $D = \exp(-\sum_i f_i \ln f_i)/S$, where $f_i = n_i^*/\sum_{j}n_j^*$ is the equilibrium frequency of species $i$. When all $S$ species are equally abundant at equilibrium, we have $D = 1$; when on the contrary one species dominates all others, we have $D = 0$. Furthermore, to avoid any artifact due to the singular behavior of the sublinear production $P(n) \sim n^k$ at low densities, we consider any species with $n_i^* < 1$ (the threshold at which $P_i = r_i$ for any value of $k$) as effectively extinct, setting $f_i = 0$ for this species. 

% Fig. XXX displays the equilibrium diversity $D$ as a function of the parameters $\mu$ and $\sigma$ characterizing the strength and heterogeneity of competitive interactions. These results are obtained by the numerical integration of \eqref{model} with $S = 50$. 

\subsection{XXX}

\section{Supplements}

\subsection{Uniform interactions}

The exact calculation. 

\subsection{Spectra of partially random matrices}

Ahmadian \emph{et al.} show that the spectrum of matrices of the form $M + LJR$, where $M$,  $L$ and $R$ are $S\times S$ deterministic matrices and $J$ is a $S\times S$ random matrix whose coefficients have mean zero and variance $\sigma^2$ is contained in the region of the complex plane 
\begin{equation}
    \textrm{Tr}[(M_zM_z^\dagger)^{-1}]\geq 1/\sigma^2 \quad \textrm{where}\ M_z = L^{-1}(zI - M)R^{-1}, 
\end{equation}
where $\textrm{Tr}$ denotes the trace of the matrix. In the special case where $L$, $R$ and $M$ are diagonal, this condition reads
\begin{equation}
    \sum_{i=1}^S\frac{(L_{i}R_{i})^2}{ \vert z - M_{i}\vert^2 }\geq 1/\sigma^2.
\end{equation}

Following Stone, we can apply this result to estimate the location of the eigenvalues of the community matrix $C$ given equilibrium abundances $\mathbf{n}^*$. Assuming random interactions with mean $\mu$ and variance $\sigma^2$, we can write $A = \mu \mathbf{1} - \mu I + J$. The community matrix in turn reads 
\begin{equation}
    C = \begin{cases}
        -D(A + I/K)\quad&\text{(GLV)}\\
        -D A - (1-k)D^{k-1}\quad&\text{(new model)} 
    \end{cases}
\end{equation}
where $D = \textrm{diag}(\mathbf{n}^*)$. For the GLV model, we take $L = -
D$, $R = I$ and $M = - D/K + \mu I$, hence eigenvalues lie within the domain
\begin{equation}
\sum_{i} \frac{(n_i^*)^2}{\vert z + n_i^*(1/K - \mu)\vert ^2}\geq 1/\sigma^2. 
\end{equation} 
This domain touches $z = 0$ (triggering an instability) whenever
\begin{equation}
    \sum_i \frac{(n_i^*)^2}{(n_i^*(1/K - \mu))^2} = \frac{S}{(1/K - \mu)^2} \geq 1/\sigma^2.
\end{equation}
This is Stone's derivation of the classical stability condition \eqref{may}. 

In the sublinear model, we have $L = -D$, $R = I$ and $M = \mu I - (1-k) D^{k-1}$. It follows that eigenvalues are contained within the domain
\begin{equation}
    \sum_{i} \frac{(n_i^*)^2}{\vert z -\mu n_i^* + (1-k) (n_i^*)^{k-1} \vert ^2}\geq 1/\sigma^2. 
\end{equation} 
As before, this domain contains $z=0$ if 
\begin{equation}
    \sum_{i} \frac{1}{\vert \mu - (1-k) (n_i^*)^{k-2} \vert ^2}\geq 1/\sigma^2. 
\end{equation} 
This is \eqref{stability-condition}. 


\printbibliography
\end{document}


